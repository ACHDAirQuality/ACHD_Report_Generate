
\newcommand\AQIPittToday{59}
\newcommand\AQIPittTom{44}
\newcommand\AQILCToday{70}
\newcommand\AQILCTom{45}
\newcommand\AQIPittTodayCate{Moderate}
\newcommand\AQIPittTomCate{Good}
\newcommand\AQILCTodayCate{Moderate}
\newcommand\AQILCTomCate{Good}
\newcommand\AQIpitttodaypolutant{PM 2.5}
\newcommand\AQIpitttomorrowpolutant{Ozone}
\newcommand\AQILCtodaypolutant{PM 2.5}
\newcommand\AQILCtomorrowpolutant{PM 2.5}
\newcommand\Discriptions{PM2.5 will remain elevated until the breaking of the inversion through 10 a.m., then levels will drop during the late morning and afternoon as the mixed layer becomes deeper. Overall PM2.5 daily averages will end up in the moderate range. The flow will turn to a more humid southwesterly one, and a shower or thundershower will develop late in the afternoon. Ozone, therefore, will fall just short of the moderate range. *** Sunday's forecast: Mostly cloudy and humid Sunday with more numerous showers and thunderstorms likely. Cloud cover and rainfall potential will keep air quality good. *** Labor Day Forecast: The upper level low will be slow to lose it's influence over the area Monday, so showers and thunderstorms have to remain in the forecast. PM2.5 and ozone levels will remain low. *** Tuesday's forecast: High pressure building back in Tuesday will bring back more sunshine with near normal temperatures. A brisk northerly flow will mean continued good air quality. }
\newcommand\ADIone{Fair - 24}
\newcommand\ADItwo{Fair - 34}
\newcommand\ADIthree{Very Poor - 3}
\newcommand\ADIfour{Very Poor - 1}
\newcommand\ADIfive{Generally Poor - 19}
\newcommand\ADIsix{Fair - 31}
\newcommand\SISone{Moderate}
\newcommand\SIStwo{--}
\newcommand\SISthree{--}
\newcommand\SISfour{--}
\newcommand\SISfive{None}
\newcommand\SISsix{--}
\newcommand\Windone{S - 7}
\newcommand\Windtwo{S - 6}
\newcommand\Windthree{S - 3}
\newcommand\Windfour{SE - 2}
\newcommand\Windfive{SE - 3}
\newcommand\Windsix{SE - 3}
\newcommand\Temp{3.4 °C}
\newcommand\Depth{115 m}
\newcommand\Time{--}
\newcommand\Scale{Moderate}
\newcommand\Inversion{No upper inversion starting below ~1000 m is reported}
\newcommand\Title{Air Quality Forecast and Dispersion Outlook \\of Allegheny County, Pennsylvania for 2022-09-03}
\newcommand\AQIPittTodayColor{FFF421}
\newcommand\AQIPittTomColor{6AFE19}
\newcommand\AQILCTodayColor{FFF421}
\newcommand\AQILCTomColor{6AFE19}
\newcommand\AQIDateToday{09/03/2022}
\newcommand\AQIDateTom{09/04/2022}
\newcommand\AQIWeekToday{Saturday}
\newcommand\AQIWeekTom{Sunday}
\newcommand\inversionmode{observation}
