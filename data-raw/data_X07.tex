
\newcommand\AQIPittToday{55}
\newcommand\AQIPittTom{41}
\newcommand\AQILCToday{69}
\newcommand\AQILCTom{43}
\newcommand\AQIPittTodayCate{Moderate}
\newcommand\AQIPittTomCate{Good}
\newcommand\AQILCTodayCate{Moderate}
\newcommand\AQILCTomCate{Good}
\newcommand\AQIpitttodaypolutant{PM 2.5}
\newcommand\AQIpitttomorrowpolutant{PM 2.5}
\newcommand\AQILCtodaypolutant{PM 2.5}
\newcommand\AQILCtomorrowpolutant{PM 2.5}
\newcommand\Discriptions{Current Conditions as of 3 PM on Wednesday:Mostly sunny skies have allowed temperatures to approach the 70-degree mark this afternoon as high pressure begins to build over the region with what is left of once Hurricane Ian finally pushing eastward.Another warm day is in store for Thursday before cooler and drier conditions arrive for Friday and through the weekend.*** Thursday's Forecast:Temperatures will once again approach the 70-degree mark on Thursday after some patchy morning fog burns off.Concentrations of fine particulate matter (PM-2.5) will rise to the higher end of the good range for most locations with some areas seeing concentrations in the moderate range on average for the day.This will be due in part to light southwesterly flow and a temperature inversion setting up during the early morning hours.The inversion should break by mid-morning with hourly concentrations improving into the afternoon hours.Concentrations will rise again slightly Thursday night, but an approaching cold front will quickly improve conditions for Friday.}
\newcommand\ADIone{Generally Poor - 16}
\newcommand\ADItwo{Generally Good - 41}
\newcommand\ADIthree{Poor - 7}
\newcommand\ADIfour{Very Poor - 3}
\newcommand\ADIfive{Fair - 31}
\newcommand\ADIsix{Generally Good - 42}
\newcommand\SISone{None}
\newcommand\SIStwo{--}
\newcommand\SISthree{--}
\newcommand\SISfour{--}
\newcommand\SISfive{None}
\newcommand\SISsix{--}
\newcommand\Windone{SW - 5}
\newcommand\Windtwo{SW - 7}
\newcommand\Windthree{SW - 5}
\newcommand\Windfour{W - 6}
\newcommand\Windfive{NW - 7}
\newcommand\Windsix{NW - 9}
\newcommand\Temp{0 °C}
\newcommand\Depth{0 m}
\newcommand\Time{--}
\newcommand\Scale{None}
\newcommand\Inversion{Yes, an upper inversion starting below ~1000 m is reported}
\newcommand\Title{Air Quality Forecast and Dispersion Outlook \\of Allegheny County, Pennsylvania for 2022-10-06}
\newcommand\AQIPittTodayColor{FFF421}
\newcommand\AQIPittTomColor{6AFE19}
\newcommand\AQILCTodayColor{FFF421}
\newcommand\AQILCTomColor{6AFE19}
\newcommand\AQIDateToday{10/06/2022}
\newcommand\AQIDateTom{10/07/2022}
\newcommand\AQIWeekToday{Thursday}
\newcommand\AQIWeekTom{Friday}
\newcommand\inversionmode{observation}
