
\newcommand\AQIPittToday{48}
\newcommand\AQIPittTom{68}
\newcommand\AQILCToday{50}
\newcommand\AQILCTom{76}
\newcommand\AQIPittTodayCate{Good}
\newcommand\AQIPittTomCate{Moderate}
\newcommand\AQILCTodayCate{Good}
\newcommand\AQILCTomCate{Moderate}
\newcommand\AQIpitttodaypolutant{PM 2.5}
\newcommand\AQIpitttomorrowpolutant{PM 2.5}
\newcommand\AQILCtodaypolutant{PM 2.5}
\newcommand\AQILCtomorrowpolutant{PM 2.5}
\newcommand\Discriptions{*** Current Conditions as of 2 PM on Friday:Dew point temperatures will continue falling through the afternoon with the arrival of a drier air mass.Temperatures are already approaching the low to mid-80s under mostly sunny skies with breezy northwesterly flow developing with good air quality.*** Saturday's Forecast:Unseasonably cool and dry weather is expected for Saturday.Temperatures are only expected to reach the upper 60s to low 70s under mostly sunny skies by the afternoon.Brisk northwesterly flow with some occasional higher gusts will keep the air mass well mixed.Ozone will be the primary pollutant with the clearing skies after some morning clouds, but concentrations will be limited to the good range.Concentrations of fine particulate matter (PM-2.5) will remain inside the good range.*** Sunday's Forecast:Similar conditions as Saturday will continue Sunday with more sunshine.Temperatures will still be unseasonably cool but will rebound slightly to reach the low to mid-70s.Good air quality will continue with the breezy northwesterly flow keeping an already clean air mass well mixed.*** Monday's Forecast:High pressure sliding southward will turn the northwesterly flow from over the weekend to more southwesterly on Monday.Dew point temperatures will begin to rise but will still be rather low for what we typically see late June.PM-2.5 concentrations will climb to the higher end of the good range.Ozone concentrations will climb higher than they did over the weekend but will remain inside the good range with afternoon clouds.Temperatures will continue to rebound and approach the upper 70s to near 80-degree mark.}
\newcommand\ADIone{Fair - 22}
\newcommand\ADItwo{Very Poor - 1}
\newcommand\ADIthree{Poor - 9}
\newcommand\ADIfour{Fair - 23}
\newcommand\ADIfive{Poor - 8}
\newcommand\ADIsix{Very Poor - 3}
\newcommand\SISone{Weak}
\newcommand\SIStwo{--}
\newcommand\SISthree{--}
\newcommand\SISfour{--}
\newcommand\SISfive{None}
\newcommand\SISsix{--}
\newcommand\Windone{W - 6}
\newcommand\Windtwo{W - 7}
\newcommand\Windthree{S - 3}
\newcommand\Windfour{SW - 6}
\newcommand\Windfive{SW - 6}
\newcommand\Windsix{W - 9}
\newcommand\Temp{2.4 °C}
\newcommand\Depth{104 m}
\newcommand\Time{--}
\newcommand\Scale{Weak}
\newcommand\Inversion{Yes, an upper inversion starting below ~1000 m is reported}
\newcommand\Title{Air Quality Forecast and Dispersion Outlook \\of Allegheny County, Pennsylvania for 2022-06-20}
\newcommand\AQIPittTodayColor{6AFE19}
\newcommand\AQIPittTomColor{FFF421}
\newcommand\AQILCTodayColor{6AFE19}
\newcommand\AQILCTomColor{FFF421}
\newcommand\AQIDateToday{06/20/2022}
\newcommand\AQIDateTom{06/21/2022}
\newcommand\AQIWeekToday{Monday}
\newcommand\AQIWeekTom{Tuesday}
\newcommand\inversionmode{observation}
