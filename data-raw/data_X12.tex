
\newcommand\AQIPittToday{47}
\newcommand\AQIPittTom{49}
\newcommand\AQILCToday{46}
\newcommand\AQILCTom{48}
\newcommand\AQIPittTodayCate{Good}
\newcommand\AQIPittTomCate{Good}
\newcommand\AQILCTodayCate{Good}
\newcommand\AQILCTomCate{Good}
\newcommand\AQIpitttodaypolutant{Ozone}
\newcommand\AQIpitttomorrowpolutant{Ozone}
\newcommand\AQILCtodaypolutant{PM 2.5}
\newcommand\AQILCtomorrowpolutant{PM 2.5}
\newcommand\Discriptions{A cold front slips through the area Saturday morning, with showers and a thunderstorm into the early afternoon, then the sun will return late in the day. PM2.5 will be highest Saturday morning, but not as high as Friday morning as the inversion will be weaker. Ozone will also improve to the good range with a good deal of clouds early in the day. *** Sunday's forecast: Sunshine most of the time Sunday with a pleasantly warm afternoon. Low humidity and a northwesterly breeze will keep overall air quality in the good range. *** Fourth of July forecast: Mostly sunny skies for the Fourth with a hot afternoon as temperatures approach the 90-degree mark. Ozone production will move maximum concentrations into the moderate range, and PM2.5 levels will again go moderate with more limited mixing. *** Tuesday's forecast: The shortened work week commences on Tuesday with an area of low pressure moving out of the Great Lakes, and bringing back the chance for showers and thunderstorms. Air quality will average in the good range. }
\newcommand\ADIone{Fair - 22}
\newcommand\ADItwo{Very Poor - 4}
\newcommand\ADIthree{Generally Good - 44}
\newcommand\ADIfour{Very Good - 117}
\newcommand\ADIfive{Poor - 8}
\newcommand\ADIsix{Very Poor - 1}
\newcommand\SISone{Slight}
\newcommand\SIStwo{--}
\newcommand\SISthree{--}
\newcommand\SISfour{--}
\newcommand\SISfive{None}
\newcommand\SISsix{--}
\newcommand\Windone{W - 9}
\newcommand\Windtwo{W - 9}
\newcommand\Windthree{N - 3}
\newcommand\Windfour{NE - 2}
\newcommand\Windfive{N - 3}
\newcommand\Windsix{N - 6}
\newcommand\Temp{0.4 °C}
\newcommand\Depth{45 m}
\newcommand\Time{--}
\newcommand\Scale{Slight}
\newcommand\Inversion{Yes, an upper inversion starting below ~1000 m is reported}
\newcommand\Title{Air Quality Forecast and Dispersion Outlook \\of Allegheny County, Pennsylvania for 2022-07-02}
\newcommand\AQIPittTodayColor{6AFE19}
\newcommand\AQIPittTomColor{6AFE19}
\newcommand\AQILCTodayColor{6AFE19}
\newcommand\AQILCTomColor{6AFE19}
\newcommand\AQIDateToday{07/02/2022}
\newcommand\AQIDateTom{07/03/2022}
\newcommand\AQIWeekToday{Saturday}
\newcommand\AQIWeekTom{Sunday}
\newcommand\inversionmode{observation}
