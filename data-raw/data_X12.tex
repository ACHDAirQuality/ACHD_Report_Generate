
\newcommand\AQIPittToday{90}
\newcommand\AQIPittTom{83}
\newcommand\AQILCToday{75}
\newcommand\AQILCTom{60}
\newcommand\AQIPittTodayCate{Moderate}
\newcommand\AQIPittTomCate{Moderate}
\newcommand\AQILCTodayCate{Moderate}
\newcommand\AQILCTomCate{Moderate}
\newcommand\AQIpitttodaypolutant{Ozone}
\newcommand\AQIpitttomorrowpolutant{Ozone}
\newcommand\AQILCtodaypolutant{PM 2.5}
\newcommand\AQILCtomorrowpolutant{PM 2.5}
\newcommand\Discriptions{Current Conditions as of 2 PM on Friday:One wave of showers and thunderstorms are developing and pushing eastward well off to our east and a second wave is now developing to the west through eastern Ohio ahead of a slow-moving cold front this afternoon.Good air quality continues once again today though a change in the pattern may end the streak of good air quality days late this weekend.*** Saturday's Forecast:A slow moving cold front will push to the east on Saturday.Lingering showers are possible during the early morning hours prior to the frontal passage.Mostly cloudy skies to start the- morning hours will give way to increasing sunshine by the afternoon.Even with some afternoon sun, ozone concentrations will remain at levels inside the good range.Breezy westerly flow behind the front will help to keep concentrations of fine particulate matter (PM-2.5) inside the good range as well.Highs will reach the low to mid-70s.*** Sunday's Forecast:On Sunday, high pressure will begin to take control of the next several days of weather.Light southerly winds will provide limited mixing.Temperatures will climb into the low 80s under mostly sunny skies.Both ozone and PM-2.5 concentrations rise into the moderate range.*** Monday's Forecast:The positioning of the area of high pressure will be key for air quality conditions on Monday.It currently appears that the flow will become more of a southwesterly breeze with dry conditions continuing under sunny skies.Temperatures will reach the upper 80s to near 90 degrees.Under these conditions, ozone concentrations would likely climb to levels in the high end of the moderate range.Development of the positioning of the high will be monitored over the weekend with updates to the forecast as necessary should conditions suggest that ozone concentrations would climb higher than moderate levels and reach Code Orange.Concentrations of PM-2.5 are expected to be at levels in the moderate range.}
\newcommand\ADIone{Very Poor - 3}
\newcommand\ADItwo{Very Poor - 2}
\newcommand\ADIthree{Good - 73}
\newcommand\ADIfour{Good - 81}
\newcommand\ADIfive{Very Poor - 4}
\newcommand\ADIsix{Very Poor - 2}
\newcommand\SISone{Strong}
\newcommand\SIStwo{--}
\newcommand\SISthree{--}
\newcommand\SISfour{--}
\newcommand\SISfive{None}
\newcommand\SISsix{--}
\newcommand\Windone{SW - 7}
\newcommand\Windtwo{SW - 7}
\newcommand\Windthree{S - 3}
\newcommand\Windfour{SW - 5}
\newcommand\Windfive{W - 8}
\newcommand\Windsix{W - 9}
\newcommand\Temp{7.2 °C}
\newcommand\Depth{153 m}
\newcommand\Time{--}
\newcommand\Scale{Strong}
\newcommand\Inversion{No upper inversion starting below ~1000 m is reported}
\newcommand\Title{Air Quality Forecast and Dispersion Outlook \\of Allegheny County, Pennsylvania for 2022-05-30}
\newcommand\AQIPittTodayColor{FFF421}
\newcommand\AQIPittTomColor{FFF421}
\newcommand\AQILCTodayColor{FFF421}
\newcommand\AQILCTomColor{FFF421}
\newcommand\AQIDateToday{05/30/2022}
\newcommand\AQIDateTom{05/31/2022}
\newcommand\AQIWeekToday{Monday}
\newcommand\AQIWeekTom{Tuesday}
\newcommand\inversionmode{observation}
